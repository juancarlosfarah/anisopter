\documentclass[a4paper,11pt]{article}
\usepackage[margin=2cm]{geometry}

\usepackage[nodayofweek]{datetime}
\usepackage{cite}
\usepackage{graphicx}
\longdate

\usepackage{hyperref}
\usepackage{fancyhdr}
\pagestyle{fancyplain}
\fancyhf{}
\lhead{\fancyplain{}{M.Sc.\ Group Project Report}}
\rhead{\fancyplain{}{\today}}
\cfoot{\fancyplain{}{\thepage}}


\title{Implementation of attentional bistability of the dragonfly visual neurons in an intelligent biomimetic agent\\\Large{--- Final Report ---}}
\author{Juan Carlos Farah, Panagiotis Almpouras, Ioannis Kasidakis, Erik Grabljevec, Christos Kaplanis\\
       \{jcf214, pa512, ik311, eg1114, ck2714\}@doc.ic.ac.uk\\ \\
       \small{Supervisors: Professor Murray Shanahan, Zafeirios Fountas, Pedro Mediano}\\
       \small{Course: CO530/533, Imperial College London}
}

\begin{document}
\maketitle

\section{Final Product}

From beginning till completion of this project a lot of hard work was put on it. The complex nature of 
the field of neuroscience as a whole significantly increased the overall level of difficulty of the 
project. Our goal was not just to replicate the neural processes occurring when the dragonfly preys 
but to create a tool that models them in the most realistic way possible and that provides helpful 
metrics for the analysis of those processes.\par

For most parts we accurately estimated the level of difficulty so that we can properly allocate 
resources on. Some however proved more challenging than we originally expected. This section 
summarises the goals that we met, the targets that we could only partially complete, the reasons
why we could not fully address some of the issues that arose during the project as well as further
development that could prove fruitful in the future.

\subsection{Goals Met}

As can be seen in the Specification section of this report we managed to fully complete most of the 
targets that we set. In fact, we managed to complete all the targets set not only as minimum 
requirements (Stage 1) but also for the expected implementation (Stage 2). More specifically, we 
managed to:

\begin{enumerate}

 \item{Create an animation tool that allows the generation of input for visual processing. (Minimum 
Requirement)}

\item{Build a model for the ESTMD neuron present between the retina and the actual CSTMD 
neurons of a real dragonfly. The function of this neuron is, given visual input, to isolate small targets 
from a potentially noisy background. (Minimum Requirement)}  

\item{Build a layer of pattern recognition neurons that can be trained in an unsupervised manner to 
identify spike patterns within a noisy background. (Minimum Requirement)}

\item{Integrate the visual processing and pattern recognition system to detect patterns within the
CSTMD output and add a simple action selection mechanism. (Minimum Requirement)}

\item{Develop a web client to provide an interface and key metrics for each module and for the 
system as a whole.  (Expected Implementation)}

\item{Create an animation for the dragonfly agent to visualise the results of a simulation. (Expected 
Implementation)}

\item{Enhance the action selection mechanism to control the agent within the environment 
(Expected Implementation)}

\item{Improve the usability and features of the web client (Possible Exentions)}

\end{enumerate}

\subsection{Partially completed tasks}

The task that proved to be the most challenging aspect of the project was the development of the 
CSTMD neuron. It is observed that the CSTMD neuron when provided with multiple stimuli, it spikes 
depending only on one of them thus providing a selection mechanism \cite{?}. In the beginning of 
the project we were provided with third party code that replicated that process but was tested only 
for a specific type of input. With the visual input that was key to our project however the CSTMD 
neuron did not show the expected behaviour and it failed to provide a robust selection mechanism. 
Despite our constant efforts to make it work, the time constraints of this project along with the high 
complexity of the other modules made it infeasible to fully meet this target. Therefore the target:\\

\noindent

Achieve CSTMD target selection through experimentation with various parameters and connections 
with the ESTMD neuron (Possible Extensions)
was only partially completed.

\subsection{Non-implemented tasks}

Of all the goals that were set there was only one we did not get the chance to work on:

\noindent
Implement the agent in a quadcopter drone (Possible Extensions).
The idea behind this goal was to integrate the completed agent to an actual quadcopter with 
embedded cameras that would allow it to react in real time to stimuli. That however would require
very fast completion of the whole system or extended period of time to work on this project. What is 
more, the required processing of the visual input within each simulation is so computationally 
expensive that made it impossible to implement a real time system with the existing hardware. 
Therefore this goal could not be met.

\subsection{Future Development}

This project can definitely motivate future development.  Even as an individual module, the CSTMD 
neuron could prove very useful to the science community. We may have not been able to model its 
behaviour properly, we did however observe that the existing tools regarding this module are 
insufficient and allow for plenty of space for improvement. What is more, if the system gets fully 
implemented, integrating it to a quadcopter with powerful processors could lead to numerous 
interesting applications.

\subsection{Project Evaluation}

All in all, we feel that we can consider this project a successful one. That is not only because we 
manage to meet most of the goals that we set despite their high level of difficulty but also because it 
was a great experience that every member of this team regards as valuable. There was a steep 
learning curve throughout the project and every member got the chance to familiarise themselves 
with several tools that have both generic and specific(???) applications. Team work was key to the 
success of this project and we are very satisfied that we managed to properly cooperate in a flexible 
and efficient way.




\end{document}

