\documentclass[a4paper,11pt]{article}

\begin{document}

\section{Feasibility and Risks}

Each of the components of our minimum and expected requirements is being structured according to relevant published research that serves as its starting point and general heuristic. This allows us to outline the aforementioned specifications so that the core of our project is highly feasible from the onset. Stage 1A bases itself on [REFS], Stage 1B on [stdp1, stdp2] and Stage 1C on [REF]. Moreover, our approach for Stage 2 will stem from [REF].

Nevertheless, there are a number of risks that could influence the outcome of our project. Firstly, it is important 
to highlight that our team is not fully acquainted with a number of computational neurodynamics concepts. Secondly, following modular development, each of the components our project is being developed independently. This introduces the possibility of incompatibility between each of the parts. This is exemplified by the complex connections needed between the visual pre-processing, CSTMD1 and pattern recognition neurons, and furthermore, between these and the action-selection mechanism. We are continually addressing these respective risks by going through recommended readings provided by our supervisors and maximising communication channels, code integrity and visibility as specified in our project methodology.

\end{document}

