\documentclass[a4paper,11pt]{article}

\begin{document}

\section{Development Methodology}

\subsection{Project Management}

\subsubsection{Agile Scrum}

Given the complexity of our project and its division into clear working stages, we chose to apply the Scrum Methodology of Agile Development as our project management framework.

\subsubsection{Sprints}
The project has been divided into seven two-week sprints starting on 4 February 2015. The first five sprints have been designed to complete the minimum and expected goals, with the last two reserved to refine results, prepare the presentation and optionally implement extensions. Sprint Review meetings will be held every two Wednesdays in the presence of co-supervisors Zafeirios Fountas and Pedro Mart�nez-Mediano, followed directly by the Sprint Planning meeting for the upcoming sprint. Sprints are being managed using Trello, with each task given an estimate of one to eight hours and a priority ranging from one (Trivial) to five (Blocker).

\subsubsection{Stand Ups}
Every weekday the team meets for a brief stand up meeting capped at 15 minutes. During this time, the state of the current sprint is assessed, with each team member outlining what he achieved on the previous day and what he will be working on before the next standup. Any possible bottlenecks or roadblocks identified during these stand ups are later addressed by the Team Leader.

\subsection{Collaboration Tools}

\subsubsection{Communication Channels}
All of our communication is conducted using Slack, which allows team members visibility to all aspects of the project. Relevant information is shared, discussed and classified using Slack's channeled feeds, with a dedicated channel assigned to each sprint.

\subsubsection{Version Control}
We are using Git as our version control system, hosting our remote repository on GitLab, which integrates with our communication and project management tools. By taking advantage of web hooks, pushes to the repository can trigger automatic code reviews and unit tests, ensuring code integrity and keeping all team members and supervisors up to date with progress.

\end{document}

