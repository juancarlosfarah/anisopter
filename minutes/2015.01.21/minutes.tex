\documentclass{article}

\usepackage{paralist}
\usepackage[colorlinks=true]{hyperref}
\usepackage{circuitikz}
\usepackage{textcomp}
\usepackage{mathcomp}
\usepackage{fullpage}
\usepackage{bm}
\usepackage{amsmath,amssymb,amsthm,enumitem}

\title{Group Project Minutes}
\author{Logged by: Ioannis Kasidakis}
\date{21 January 2015}

\begin{document}

\maketitle

\section*{Attendance}
\begin{compactenum}
\item Juan Carlos Farah (JCF)
\item Christos Kaplanis (CK)
\item Erik Grabljevec (EG)
\item Panagiotis Almpouras (PA)
\item Ioannis Kasidakis (IK)
\end{compactenum}

\section*{Summary}
This meeting was mainly focused on discussing questions we had after reading all the relevant papers as well as what should be the next steps. A joint meeting took place after which, separate group meetings were conducted with each relevant supervisor to discuss in more detail about the specifics of the upcoming tasks.

\section*{Joint Meeting}

{\addtolength{\leftskip}{3mm}
\subsection*{Takeaway Points}
}
\begin{compactenum}
\item The first report will be delivered on 6th February.
\item Stage 1 of the project needs to be finished by 13th February the latest.
\item Pattern recognition is estimated to take about 3 weeks.
\item With regards to the requirements of the project:
	\begin{compactenum}
	\item The worst-case and realistic scenario is to complete stages 1 and 2 with different levels of completion respectively.
	\item The ideal scenario is to complete all 3 stages.
	\end{compactenum}
\item As discussed, dragonfly neurons will do the premature selection of the target but the proper selection will be done by us.
\end{compactenum}

{\addtolength{\leftskip}{3mm}
\subsection*{Action Points}
}
\begin{compactenum}
\item We need to decide how many neurons we are going to use.
\item We need to decide how many inputs we are going to need.
\item We need to decide about the robustness of the patterns.
\end{compactenum}

\clearpage

\section*{Dragonfly Neuron Meeting}

{\addtolength{\leftskip}{3mm}
\subsection*{Takeaway Points}
}
\begin{compactenum}
\item According to the current Zafeirios's neuron implementation, each neuron has shape similar to a polygon. Its dendrites are randomly formed in a biologically plausible way and the general morphology of the neuron is replicated by the paper \textit{Local and large ­range inhibition in feature detection. DM Bolzon, K Nordström and DC O'Carroll}.
\item The first part is to translate the input from an event camera/webcam to an input which can be used by the neuron implementation. The stages for this part are:
	\begin{compactenum}
	\item Split the video of the event camera/webcam to separate frames.
	\item Reduce the definition of the frames (probably to 32x32 pixels).
	\item Convert the frames to grayscale pictures.
	\item For each pixel and for each frame, find the difference in grayscale color with its next frame.
	\item Each pixel will be considered to feed the results to a vision neuron.
	\item Create random Poisson events with numpy in order to determine whether each pixel will result in the respective vision neuron to be fired.
	\item Create a matrix/array for each pixel with 1s and 0s(1 if the neuron fired a spike, 0 otherwise).
	\item Feed this input to the neuron model.
	\end{compactenum}
\item The second part is to decide about the structure of the model e.g the number of neurons we are going to use and  the number of synapses.

\item After the completion of both parts, we are going to run tests in order to determine whether the model is accurate and possibly change any parameters if required. The aim is to create a model which can ideally replicate or at least approximate the experiment of the paper \textit{Local and large ­range inhibition in feature detection. DM Bolzon, K Nordström and DC O'Carroll}.
\end{compactenum}

{\addtolength{\leftskip}{3mm}
\subsection*{Action Points}
}
\begin{compactenum}
\item Search whether there is any Python library which can help with the implementation of the first part.
\item Start the implementation of the first part.
\end{compactenum}

\section*{Action Selection Meeting}

{\addtolength{\leftskip}{3mm}
\subsection*{Takeaway Points}
}
\begin{compactenum}
\item
\end{compactenum}

{\addtolength{\leftskip}{3mm}
\subsection*{Action Points}
}
\begin{compactenum}
\item
\end{compactenum}

\end{document}