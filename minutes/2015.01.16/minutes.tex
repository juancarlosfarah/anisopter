\documentclass{article}

\usepackage{paralist}
\usepackage[colorlinks=true]{hyperref}
\usepackage{circuitikz}
\usepackage{textcomp}
\usepackage{mathcomp}
\usepackage{fullpage}
\usepackage{bm}
\usepackage{amsmath,amssymb,amsthm,enumitem}

\title{Group Project Minutes}
\author{Logged by: Juan Carlos Farah}
\date{16 January 2015}

\begin{document}

\maketitle

\section*{Attendance}
\begin{compactenum}
\item Juan Carlos Farah (JCF)
\item Christos Kaplanis (CK)
\item Erik Grabljevec (EG)
\item Panagiotis Almpouras (PA)
\item Ioannis Kasidakis (IK)
\end{compactenum}

\section*{Summary}
This meeting was mainly focused on discussing the way that the group would tackle the project's administrative tasks and get the ball rolling with preliminary readings and installation of software needed.

\section*{Takeaway Points}
\begin{compactenum}
\item Weekly team meetings scheduled for Wednesdays at 1400.
\item Weekly meeting notes will be recorded in LaTeX and added to the repository.
\item Python style will be based on the \href{https://google-styleguide.googlecode.com/svn/trunk/pyguide.html}{Google Python Style Guide} and \href{https://www.python.org/dev/peps/pep-0008/}{PEP 8}.
\item Folder structure for meeting notes will be as follows:
	\begin{compactenum}
	\item In repository's root folder there is a \textbf{minutes} folder.
	\item For each meeting there is a folder with the name following YYYY.MM.DD format, e.g. \textbf{2015.01.16}.
	\item If there are two meetings the same day, just add suffix -a, -b, etc., e.g. \textbf{2015.01.16-b}.
	\item Inside each of these folders you should find a \textbf{minutes.tex} file, and a \textbf{minutes.pdf} compiled filed. All other files will be in the .gitignore, so unless you compiled that .tex file manually, you shouldn't see them.
	\end{compactenum}
\item The \textbf{minutes.pdf} files will be also made available through the Google Drive.
\item Any action points in the minutes should be added to Trello by the logger.
\end{compactenum}


\section*{Action Points}
\begin{compactenum}
\item Team to read all papers provided and share thoughts via email.
\item Team to install NEURON in the systems they will be using.
\item Team to clone the repository.
\item CK and IK to read the code provided.
\item JCF to create LaTeX template for meeting notes.
\item JCF to create add unnecessary .gitignore 
\end{compactenum}

\end{document}